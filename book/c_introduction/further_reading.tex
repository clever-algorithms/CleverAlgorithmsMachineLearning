% The Clever Algorithms Project: http://www.CleverAlgorithms.com
% (c) Copyright 2011 Jason Brownlee. Some Rights Reserved. 
% This work is licensed under a Creative Commons Attribution-Noncommercial-Share Alike 2.5 Australia License.

% Further Reading
\section{Further Reading} 
\label{intro:furtherreading}
\index{Further Reading}
% overview
This book is not an introduction to Machine Learning or related sub-fields, nor is it a field guide for a specific class of algorithms. This section provides some pointers to selected books and articles for those readers seeking a deeper understanding of the fields of study to which the Clever Algorithms described in this book belong.

% Artificial Intelligence
\subsection{Artificial Intelligence}
\index{Artificial Intelligence!References}
Artificial Intelligence is large field of study and many excellent texts have been written to introduce the subject. Russell and Novig's ``\emph{Artificial Intelligence: A Modern Approach}'' is an excellent introductory text providing a broad and deep review of what the field has to offer and is useful for students and practitioners alike \cite{Russell2009}. Luger and Stubblefield's ``\emph{Artificial Intelligence: Structures and Strategies for Complex Problem Solving}'' is also an excellent reference text, providing a more empirical approach to the field \cite{Luger1993}.

% Machine Learning
\subsection{Machine Learning}
\index{Machine Learning!References}
Machine Learning has changed a lot over the last 20 years. 
Tom Mitchell's ``\emph{Machine Learning}'' is classic text providing an early shaping of the field \cite{Mitchell1997}.
Duda et~al.\ also provide a classic reference text in ``\emph{Pattern Classification}'' that provides a broader perspective of machine learning \cite{Duda2001}.

Bishop's ``\emph{Pattern Recognition and Machine Learning}'' provides an excellent theoretical study of modern machine learning methods \cite{Bishop2007}.
Hastie et~al.\ text ``\emph{The Elements of Statistical Learning: Data Mining, Inference, and Prediction}'' provides a modern perspective on the field, aggressively seeking sound statistically theoretical understanding of many modern methods \cite{Hastie2009}.

% Data Mining
\subsection{Data Mining}
\index{Data Mining!References}
Data Mining may be taken as a filed concerned with applied Machine Learning. 
Witten and Frank provide an excellent introduction into the field of Data Mining with some practical examples in their renowned open source Data Mining WEKA software \cite{Witten2011}.
% TODO another ref!

% R
\subsection{R}
\index{R!References}
There are many excellent books on R and almost all of them are free and can be found on the R website at \url{http://www.r-project.org}.

Two non-free excellent introductions to R that should be considered by a new-comer to R include Crawley's ``\emph{The R Book}'' which provides an extensive reference \cite{Crawley2007} and Matloff's ``\emph{The Art of R Programming: A Tour of Statistical Software Design}'' which provides a focus on programing in R \cite{Matloff2011}.

Finally, the true power in R is in the extension packages on CRAN. The best way to get started with CRAN is to look at some `views' that group like-packages under a common theme. Take a look at the \emph{Machine Learning and Statistical Learning}, \emph{Optimization and Mathematical Programming}, and \emph{Cluster Analysis and Finite Mixture Models} views at the CRAN views webpage \url{http://cran.r-project.org/web/views}.

% The Clever Algorithms Project: http://www.CleverAlgorithms.com
% (c) Copyright 2011 Jason Brownlee. Some Rights Reserved. 
% This work is licensed under a Creative Commons Attribution-Noncommercial-Share Alike 2.5 Australia License.

% Name
% The algorithm name defines the canonical name used to refer to the technique, in addition to common aliases, abbreviations, and acronyms. The name is used in terms of the heading and sub-headings of an algorithm description.
\section{Stepwise Linear Regression} 
\label{sec:stepwise}
\index{Stepwise Linear Regression}

% other names
% What is the canonical name and common aliases for a technique?
% What are the common abbreviations and acronyms for a technique?
\emph{Stepwise Linear Regression, Stepwise Selection}

% Taxonomy: Lineage and locality
% The algorithm taxonomy defines where a techniques fits into the field, both the specific subfields of Computational Intelligence and Biologically Inspired Computation as well as the broader field of Artificial Intelligence. The taxonomy also provides a context for determining the relation- ships between algorithms. The taxonomy may be described in terms of a series of relationship statements or pictorially as a venn diagram or a graph with hierarchical structure.
\subsection{Taxonomy}
% To what fields of study does a technique belong?
% What are the closely related approaches to a technique?
Stepwise Regression is a model selection method for selecting a regression model. It is typically applied to Linear Regression and Generalized Linear Regression Models.

% Strategy: Problem solving plan
% The strategy is an abstract description of the computational model. The strategy describes the information processing actions a technique shall take in order to achieve an objective. The strategy provides a logical separation between a computational realization (procedure) and a analogous system (metaphor). A given problem solving strategy may be realized as one of a number specific algorithms or problem solving systems. The strategy description is textual using information processing and algorithmic terminology.
\subsection{Strategy}
% What is the information processing objective of a technique?
% What is a techniques plan of action?

used for feature selection
greedy algorithm - adds the best feature or removes the worst feature each iteration
have to decide when to stop the algorithm - stopping condition - typically done by cross validation
some criteria can be optimized

see \url{http://www.stata.com/support/faqs/stat/stepwise.html}
good example \url{http://cran.r-project.org/doc/contrib/Faraway-PRA.pdf}

% help me use this technique
\subsection{Overview}

% what it is good at
\subsubsection{Features}

\begin{itemize}
	\item Simple method for feature selection.
	\item One of a variety of selection criteria can be used, common examples include the Akaike Information Criterion (AIC), and the Bayes Information Criterion (BIC).
\end{itemize}

% what it is not good at
\subsubsection{Limitations}

\begin{itemize}
	\item Stepwise models are a deprecated method as they have a high chance of selecting the wrong attributes and being optimistic of the model they prepare.
\end{itemize}



% sample script in R
\subsection{Code Listing}
% listing
Listing~\ref{stats_stepwise_linear_regression} provides a code listing of Stepwise Linear Regression method in R to find the relevant features and the line of best fit for those features.
% algorithm and package
The example uses the {lm()} function and the and \texttt{step()} in the \texttt{stats} core package which are responsible for fitting linear models and performing stepwise selection respectively. The \texttt{step()} function uses a Akaike Information Criterion as the model evaluation criteria.

% problem
The test problem is a four-dimensional dataset of 50 samples, where the x-values are drawn from a uniformly random distribution $x \in [0,10]$ and \texttt{y} values are dependent on the \texttt{x} value plus a value drawn from a normally random distribution with a mean of 0 and a standard deviation of 1. The features \texttt{a} and \texttt{b} are random independent variables that have no relevant interaction x and y. In this example, \texttt{y} is considered the dependent variable and x the single relevant independent. The stepwise method is expected to discount \texttt{a} and \texttt{b} and select a linear model for \texttt{y} given \texttt{x}.

% classification example?

\lstinputlisting[firstline=7,language=r,caption={Example of Stepwise Linear Regression in R using the \texttt{lm()} and \texttt{step()} functions in the \texttt{stats} core package.}, label=stats_stepwise_linear_regression]{../src/algorithms/regression/stats_stepwise_linear_regression.R}

% other packages
Other packages that provide stepwise selection include the \texttt{stepAIC} function of the \texttt{MASS} package. The \texttt{leaps} package does not provide stepwise selection, although does provides a related selection method called Regression Subset Selection that uses a branch and bound search.


% References: Deeper understanding
% The references element description includes a listing of both primary sources of information about the technique as well as useful introductory sources for novices to gain a deeper understanding of the theory and application of the technique. The description consists of hand-selected reference material including books, peer reviewed conference papers, journal articles, and potentially websites. A bullet-pointed structure is suggested.
\subsection{References}
% What are the primary sources for a technique?
% What are the suggested reference sources for learning more about a technique?

% primary sources
\subsubsection{Primary Sources}


Great early overview of the approaches (forward and backward selection) \cite{Hocking1976}.

% more info
\subsubsection{More Information}

arguments against it in modern analysis
Overview of why using stepwise regression is a bad idea \cite{Whittingham2006}.
\cite{Mundry2009}





% END

% The Clever Algorithms Project: http://www.CleverAlgorithms.com
% (c) Copyright 2011 Jason Brownlee. Some Rights Reserved. 
% This work is licensed under a Creative Commons Attribution-Noncommercial-Share Alike 2.5 Australia License.

% Name
% The algorithm name defines the canonical name used to refer to the technique, in addition to common aliases, abbreviations, and acronyms. The name is used in terms of the heading and sub-headings of an algorithm description.
\section{L-BFGS} 
\label{sec:lbfgs}
\index{Limited Memory Broyden-Fletcher-Goldfarb-Shanno}
\index{L-BFGS}
\index{L-BFGS-B}

% other names
% What is the canonical name and common aliases for a technique?
% What are the common abbreviations and acronyms for a technique?
\emph{Limited Memory Broyden-Fletcher-Goldfarb-Shanno, L-BFGS, LM-BFGS.}

% Taxonomy: Lineage and locality
\subsection{Taxonomy}
% To what fields of study does a technique belong?
% What are the closely related approaches to a technique?
L-BFGS is an optimization method for multidimensional nonlinear unconstrained functions.
L-BFGS-B is an extension of L-BFGS that adds a box-constraint (function bounds).

% Strategy: Problem solving plan
% The strategy is an abstract description of the computational model. The strategy describes the information processing actions a technique shall take in order to achieve an objective. The strategy provides a logical separation between a computational realization (procedure) and a analogous system (metaphor). A given problem solving strategy may be realized as one of a number specific algorithms or problem solving systems. The strategy description is textual using information processing and algorithmic terminology.
\subsection{Strategy}
% What is the information processing objective of a technique?
% What is a techniques plan of action?

Don't need to select a learning rate.

% help me use this technique
\subsection{Overview}

% what it is good at
\subsubsection{Features}

\begin{itemize}
	\item Uses less memory than BFGS.
	\item Intended for problems where the Hessian matrix is hard to obtain.
	\item Intended for large dense problems.
\end{itemize}

% what it is not good at
\subsubsection{Limitations}

\begin{itemize}
	\item 
\end{itemize}


% sample script in R
\subsection{Code Listing}
% listing
Listing~\ref{stats_lbfgs} provides a code listing of the L-BFGS-B method in R solving a two-dimensional nonlinear unconstrained optimization function.
% algorithm and package
The example uses the {optim()} function in the \texttt{stats} core package configured to use the L-BFGS-B method. The algorithm implementation is well described by Zhu, et al. \cite{Zhu1997}. For more information on the function, type \texttt{?optim}.
% problem
The test problem is the Rosenbrock function in two-dimensions where the optimum is at $x=1, y=1$. The starting position for the algorithm is taken as a random point $x,y \in [-3,3]$. The gradient for this function is specified and is used by the optimization method.

% TODO consider capturing algorithm progression (using par)

\lstinputlisting[firstline=7,language=r,caption={Example of L-BFGS-B in R using the \texttt{optim()} function in the \texttt{stats} core package.}, label=stats_lbfgs]{../src/algorithms/optimization/stats_lbfgs.R}

% other packages ???


% References: Deeper understanding
% The references element description includes a listing of both primary sources of information about the technique as well as useful introductory sources for novices to gain a deeper understanding of the theory and application of the technique. The description consists of hand-selected reference material including books, peer reviewed conference papers, journal articles, and potentially websites. A bullet-pointed structure is suggested.
\subsection{References}
% What are the primary sources for a technique?
% What are the suggested reference sources for learning more about a technique?

% primary sources
\subsubsection{Primary Sources}


% more info
\subsubsection{More Information}



% END

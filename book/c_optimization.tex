% The Clever Algorithms Project: http://www.CleverAlgorithms.com
% (c) Copyright 2011 Jason Brownlee. Some Rights Reserved. 
% This work is licensed under a Creative Commons Attribution-Noncommercial-Share Alike 2.5 Australia License.

% This is a chapter

\renewcommand{\bibsection}{\subsection{\bibname}}
\begin{bibunit}

\chapter{Optimization}
\label{ch:optimization}
\index{Optimization}
\index{Function Optimization}

\section{Overview}
This chapter describes unconstrained Optimization methods that common in the field of Machine Learning.

% Types of Optimization Algorithms
\subsection{Taxonomy}
index{Function Optimization!Taxonomy}
The field of function optimization is mature, traditionally referred to as Mathematical Programming. Optimization forms the core of most Machine Learning methods, and many explicitly use a classical optimization method. For the purposes of our understanding of optimization in the context of Machine Learning, we can consider the following types of optimization algorithms:

\index{Direct Search}
\index{Stochastic Direct Search}
\index{Convex Optimization}
\index{Nonlinear Optimization}
\begin{itemize}
	\index{Line Search}
	\index{Pattern Search}
	\index{Rosenbrock's Method}
	\index{Nelder-Mead Search}
	\index{Powell's Conjugate Direction Method}
	\item \textbf{Direct Search}: These are methods that only sample the function to decide how to progress the search. These are generally inefficient compared to derivative based approaches, but have the advantage of working on functions where a derivative cannot be computed or cannot be computed easily, such as noisy or discontinuous functions. Any gradient information is inferred from direct samples. The name `Direct Search' was proposed by Hooke and Jeeves \cite{Hooke1961} and a modern review of such methods can be seen in Lewis et~al. \cite{Lewis2000}. Examples of Direct Search algorithms include Line Search methods, Pattern Search, Rosenbrock's Method, Nelder-Mead Search, and Powell's Conjugate Direction Method.

	\index{Simulated Annealing}
	\index{Genetic Algorithm}
	\index{Particle Swarm Optimization}
	\item \textbf{Stochastic Direct Search}: These are methods that are stochastic processes that sample the objective function in a direct manner. Examples include Simulated Annealing, Genetic Algorithms, and Particle Swarm Optimization.

	\index{Gradient Descent}
	\index{Steepest Descent}
	\index{Conjugate Gradient Descent}
	\item \textbf{First-order Derivative}: These are methods that require the calculation of the first partial derivative (function gradient) of the objective function that is being optimized. For those problems where a gradient can be computed directly, these methods are generally more efficient (converge faster) than Direct Search methods. Examples include Gradient Descent, Steepest Descent, and Conjugate Gradient Descent.

	\index{Newton's Method}
	\index{Newton-Raphson Method}
	\index{Gauss-Newton Method}
	\index{Levenberg-Marquardt Method}
	\index{quasi-Newton Methods}
	\index{BFGS}
	\index{DFP Method}
	\index{Broyden's Method}
	\index{SR1 Method}
	\item \textbf{Second-order Derivative}: These are methods that require the calculation of a Hessian matrix of second partial derivatives (or an approximation thereof) in addition to the first-order gradient. Some second-order methods also require the storage of a matrix which must be maintained during the search procedure. Compared to first-order methods, these methods have a grater computational and/or memory expense although are generally more efficient (converge faster) given the grater amount of information available about the function. Examples include Newton's Method (Newton-Raphson Method), Gauss-Newton Method, Levenberg-Marquardt Method, and quasi-Newton Methods (Hessian approximation methods) such as BFGS, DFP Method, Broyden's method and the SR1 Method.
\end{itemize}

% Nomenclature
\subsection{Nomenclature}
index{Function Optimization!Nomenclature}
% conceputlaization
Function optimization is commonly conceptualized geometrically, where the parameters of the function define an $n$-dimensional search space and the role of the algorithm is to locate the desired point within that search space.

% optima
A problem may be referred to as an objective function, cost function, utility function, or loss function and it may be minimization or maximization. One may talk about the objective of the function in the abstract as a function extremum or function optimum. 
% shape
A function have one or more optima, and the shape of the functions response surface (the result of the function given parameter values) is referred to as being unimodal or multimodal respectively. The single optimum of a unimodal function means that the local optimum is the global optimum. Highly-nonlinear and multimodal problems can have many optima, meaning that some optimization methods can get caught in a local optimum and fail to fund the global optimum.

% convex
Many Machine Learning methods have been defined with the optimization of a convex function as the core problem. For our purposes, a convex function is unimodal and is a type of function that looks like a bowl or basin shape when in two-dimensions.

% references
\subsection{Further Reading}
\index{Function Optimization!Further Reading}
% general
There are many texts on modern optimization methods. Some good general reference texts include Boyd and Vandenberghe  who provide a comprehensive introduction into the field of convex function optimization \cite{Boyd2004}, and Griva et~al. who provide a broader walk of the field of linear and non-linear optimization \cite{Griva2009}. Nocedal and Wright also provide a thorough text on numerical optimization methods \cite{Nocedal1999}.
% for machine learning
Reed et~al. provide an excellent overview of modern optimization methods in the context of their adaptation for use in Artificial Neural Networks \cite{Reed1998} (Chapter 10).
% R 
Braun et~al. provide a gentle introduction to optimization algorithms in R with some worked examples \cite{Braun2007} (Chapter 7).

\putbib
\end{bibunit}

\newpage\begin{bibunit}% The Clever Algorithms Project: http://www.CleverAlgorithms.com
% (c) Copyright 2011 Jason Brownlee. Some Rights Reserved. 
% This work is licensed under a Creative Commons Attribution-Noncommercial-Share Alike 2.5 Australia License.

% Name
% The algorithm name defines the canonical name used to refer to the technique, in addition to common aliases, abbreviations, and acronyms. The name is used in terms of the heading and sub-headings of an algorithm description.
\section{Golden Section Search} 
\label{sec:golden_section_search}
\index{Golden Section Search}
\index{Golden Mean Search}

% other names
% What is the canonical name and common aliases for a technique?
% What are the common abbreviations and acronyms for a technique?
\emph{Golden Section Search, Golden Mean Search.}

% Taxonomy: Lineage and locality
\subsection{Taxonomy}
\index{Line Search}
\index{Direct Search}
\index{Pattern Search}
\index{Global Optimization}
% To what fields of study does a technique belong?
Golden Section Search is a Line Search method for Global Optimization in one-dimension. It is a Direct Search (Pattern Search) method as it samples the function to approximate a derivative rather than computing it directly.
% What are the closely related approaches to a technique? 
The Golden Section Search is related to pattern searches of discrete ordered lists such as the Binary Search and the  Fibonacci Search. It is related to other Line Search algorithms such as Brent's Method and more generally to other direct search optimization methods such as Gradient Descent.

% Strategy: Problem solving plan
% The strategy is an abstract description of the computational model. The strategy describes the information processing actions a technique shall take in order to achieve an objective. The strategy provides a logical separation between a computational realization (procedure) and a analogous system (metaphor). A given problem solving strategy may be realized as one of a number specific algorithms or problem solving systems. The strategy description is textual using information processing and algorithmic terminology.
\subsection{Strategy}
% What is the information processing objective of a technique?
The information processing objective of the method is to locate the extremum of a function.
% What is a techniques plan of action?
It does this by directly sampling the function using a pattern of three points. The points form the brackets on the search, the first and the last the current bounds of the search, and the third point that partitions the intervening space. The partitioning point is selected so that the ratio between the larger partition and the whole interval is the same as the ratio of the larger partition to the small partition, known as the golden ratio ($\phi$). The partitions are compared based on their function evaluation and the better performing section is selected as the new bounds on the search. The process recurses until the desired level of precision (bracketing of the optima) is obtained or the search stalls.

% Heuristics: Usage guidelines
% The heuristics element describe the commonsense, best practice, and demonstrated rules for applying and configuring a parameterized algorithm. The heuristics relate to the technical details of the techniques procedure and data structures for general classes of application (neither specific implementations not specific problem instances). The heuristics are described textually, such as a series of guidelines in a bullet-point structure.
\subsection{Heuristics}
% What are the suggested configurations for a technique?
% What are the guidelines for the application of a technique to a problem instance?

\begin{itemize}
	\item Assumes that the function is convex and unimodal specifically, that the function has a single optima and that it lies between the first two bracket points.
	\item Intended to find the extrema one-dimensional continuous functions.
	\item It was shown to be more efficient than equal-sized partition line search.
	\item The termination criteria is a specification on the minimum distance between the brackets on the optima.
	\item It can quickly locate the bracketed area of the optima but is less efficient at locating the specific optima.
	\item Once a solution of desired precision is located, it can be provided as the basis to a second search algorithm that has a faster rate of convergence.
\end{itemize}

% sample script in R
\subsection{Code Listing}
% listing
Listing~\ref{stats_golden_section_search} provides a code listing Golden Section Search method in R solving a one-dimensional nonlinear unconstrained optimization function.
% algorithm and package
The example uses the {optimize()} function in the \texttt{stats} core package which is limited to one-dimensional function. This function uses a Golden Section Search with successive Parabolic Interpolation. This combination of a linear search followed successively by Parabolic Interpolation is a common pattern for improving the overall result by leveraging the non-parametric nature of the former and the speed of convergence of the latter methods. For more information on the function, type \texttt{?optimize}.
% problem
The test problem is the basin function in one-dimension where the optimum is at $f(0)=0$ and the domain is defined as $x \in [-5,5]$. 

\lstinputlisting[firstline=7,language=r,caption={Example of Golden Section Search in R using the \texttt{optimize()} function in the \texttt{stats} core package.}, label=stats_golden_section_search]{../src/algorithms/optimization/stats_golden_section_search.R}

% References: Deeper understanding
% The references element description includes a listing of both primary sources of information about the technique as well as useful introductory sources for novices to gain a deeper understanding of the theory and application of the technique. The description consists of hand-selected reference material including books, peer reviewed conference papers, journal articles, and potentially websites. A bullet-pointed structure is suggested.
\subsection{References}
% What are the primary sources for a technique?
% What are the suggested reference sources for learning more about a technique?

% primary sources
\subsubsection{Primary Sources}
% seminal
The method was described by Kiefer in 1953 as a method for finding the maximum of a function without regularity conditions such as continuity and function derivatives \cite{Kiefer1953}.
Johnson provided an early description of the method in a technical report for RAND Corporation \cite{Johnson1955}.

% more info
\subsubsection{More Information}
% useful
Press et~al.\ provide a terse and practical introduction to the method with well-commented sample code in the C programming language \cite{Press2007}.
% early
Brent provides an early treatment of the method is his text on minimization without derivatives \cite{Brent1973}.
% text books
Many text books on scientific computing or numerical analysis include a section on Golden Section Search, for example, see Heath \cite{Heath2002}.

% END
\putbib\end{bibunit}
\newpage\begin{bibunit}% The Clever Algorithms Project: http://www.CleverAlgorithms.com
% (c) Copyright 2011 Jason Brownlee. Some Rights Reserved. 
% This work is licensed under a Creative Commons Attribution-Noncommercial-Share Alike 2.5 Australia License.

% Name
% The algorithm name defines the canonical name used to refer to the technique, in addition to common aliases, abbreviations, and acronyms. The name is used in terms of the heading and sub-headings of an algorithm description.
\section{Nelder-Mead Method} 
\label{sec:neldermead}
\index{Nelder-Mead Method}
\index{Simplex Method}

% other names
% What is the canonical name and common aliases for a technique?
% What are the common abbreviations and acronyms for a technique?
\emph{Nelder-Mead Method, Downhill Simplex Method, Simplex Method.}

% Taxonomy: Lineage and locality
\subsection{Taxonomy}
% To what fields of study does a technique belong?
% What are the closely related approaches to a technique?
Nelder-Mead Method is an optimization method.

Direct Search method.

% Strategy: Problem solving plan
% The strategy is an abstract description of the computational model. The strategy describes the information processing actions a technique shall take in order to achieve an objective. The strategy provides a logical separation between a computational realization (procedure) and a analogous system (metaphor). A given problem solving strategy may be realized as one of a number specific algorithms or problem solving systems. The strategy description is textual using information processing and algorithmic terminology.
\subsection{Strategy}
% What is the information processing objective of a technique?
% What is a techniques plan of action?


% help me use this technique
\subsection{Overview}

% what it is good at
\subsubsection{Features}

\begin{itemize}
	\item Suitable for convex response surfaces.
\end{itemize}

% what it is not good at
\subsubsection{Limitations}

\begin{itemize}
	\item An optimal value for the $\alpha$ parameter can be difficult to find.
\end{itemize}


% sample script in R
\subsection{Code Listing}
Listing~\ref{stat_nelder_mead} provides a code listing of the Nelder-Mead method in R using the \texttt{constrOptim()} function in the \texttt{stats} core package. 

% problem
The test problem is the Rosenbrock function in two-dimensions where $x_i\in[-2.048,2.048]$ and the optimum is at $x_i=1$.

\lstinputlisting[firstline=7,language=r,caption={Example of Nelder-Mead in R using the \texttt{constrOptim()} function in the \texttt{stats} core packag.}, label=stat_nelder_mead]{../src/algorithms/optimization/stat_nelder_mead.R}

% References: Deeper understanding
% The references element description includes a listing of both primary sources of information about the technique as well as useful introductory sources for novices to gain a deeper understanding of the theory and application of the technique. The description consists of hand-selected reference material including books, peer reviewed conference papers, journal articles, and potentially websites. A bullet-pointed structure is suggested.
\subsection{References}
% What are the primary sources for a technique?
% What are the suggested reference sources for learning more about a technique?

% primary sources
\subsubsection{Primary Sources}

seminal \cite{Nelder1965}

% more info
\subsubsection{More Information}




% END
\putbib\end{bibunit}
\newpage\begin{bibunit}% The Clever Algorithms Project: http://www.CleverAlgorithms.com
% (c) Copyright 2013 Jason Brownlee. Some Rights Reserved. 
% This work is licensed under a Creative Commons Attribution-Noncommercial-Share Alike 2.5 Australia License.


% Name
% The algorithm name defines the canonical name used to refer to the technique, in addition to common aliases, abbreviations, and acronyms. The name is used in terms of the heading and sub-headings of an algorithm description.
\section{Gradient Descent} 
\label{sec:gradient_descent}
\index{Gradient Descent}
\index{Gradient Ascent}

% other names
% What is the canonical name and common aliases for a technique?
% What are the common abbreviations and acronyms for a technique?
\emph{Gradient Descent, Gradient Ascent.}

% Taxonomy: Lineage and locality
\subsection{Taxonomy}
\index{Steepest Descent Method}
\index{Stochastic Gradient Descent}
\index{Online Gradient Descent}
\index{Batch Gradient Descent}
% To what fields of study does a technique belong?
Gradient Descent is a first-order derivative optimization method for unconstrained nonlinear function optimization. It is called Gradient Descent because it was envisioned for function minimization. When applied to function maximization it may be referred to as Gradient Ascent. 

% What are the closely related approaches to a technique? 
Steepest Descent Search is an extension that performs a Line Search on the line of the gradient to the locate the optimum neighboring point (optimum step or steepest step).
Batch Gradient Descent is an extension where the cost function and its derivative are computed as the summed error on a collection of training examples.
Stochastic Gradient Descent (or Online Gradient Descent) is like Batch Gradient Descent except that the cost function and derivative are computed for each training example.

% Strategy: Problem solving plan
% The strategy is an abstract description of the computational model. The strategy describes the information processing actions a technique shall take in order to achieve an objective. The strategy provides a logical separation between a computational realization (procedure) and a analogous system (metaphor). A given problem solving strategy may be realized as one of a number specific algorithms or problem solving systems. The strategy description is textual using information processing and algorithmic terminology.
\subsection{Strategy}
% What is the information processing objective of a technique?
The information processing objective of the method is to locate the extremum of a function.
% What is a techniques plan of action?
This is achieved by first selecting a starting point in the search space. For a given point in the search space, the derivative of the cost function is calculated and a new point is selected down the gradient of the functions derivative at a distance of $\alpha$ (the step size parameter) from the current point. 

% Heuristics: Usage guidelines
% The heuristics element describe the commonsense, best practice, and demonstrated rules for applying and configuring a parameterized algorithm. The heuristics relate to the technical details of the techniques procedure and data structures for general classes of application (neither specific implementations not specific problem instances). The heuristics are described textually, such as a series of guidelines in a bullet-point structure.
\subsection{Heuristics}
% What are the suggested configurations for a technique?
% What are the guidelines for the application of a technique to a problem instance?

\begin{itemize}
	\item The method is limited to finding the local optimum, which if the function is convex, is also the global optimum.
	\item It is considered inefficient and slow (linear) to converge relative to modern methods. Convergence can be slow if the gradient at the optimum flattens out (gradient goes to 0 slowly). Convergence can also be slow if the Hessian is poorly conditioned (gradient changes rapidly in some directions and slower in others).
	\item The step size ($\alpha$) may be constant, may adapt with the search, and may be maintained holistically or for each dimension.
	\item The method is sensitive to initial conditions, and as such, it can be common to repeat the search process a number of times with randomly selected initial positions.
	\item If the step size parameter ($\alpha$) is too small, the search will generally take a large number of iterations to converge, if the parameter is too large can overshoot the function's optimum.
	\item Compared to non-iterative function optimization methods, gradient descent has some relative efficiencies when it comes to scaling with the number of features (dimensions).
\end{itemize}

% sample script in R
\subsection{Code Listing}
% listing
Listing~\ref{gradient_descent} provides a code listing Gradient Descent algorithm in R solving a two-dimensional nonlinear optimization function. Figure~\ref{plot:gradient_descent_result} provides a plot of the test problem with the located minimum highlighted.

% algorithm and package
The example provides a custom written \texttt{gradient\_descent()} function for locating the minimum of a two-dimensional function.
% problem
The test problem is the basin function in two-dimensions where the optimum is at $f(0)=0$ and the domain is defined as $x \in [-3,3]$. 

\lstinputlisting[firstline=7,language=r,caption={Example of Gradient Descent in R using a custom function.}, label=gradient_descent]{../src/algorithms/optimization/gradient_descent.R}

\begin{figure}[htp]
\centering
\includegraphics[scale=0.45]{book/a_optimization/gradient_descent_result.png}
\caption{Contour plot of the basin function with the located minimum highlighted.}
\label{plot:gradient_descent_result}
\end{figure}

% other packages
Another other packages that provides an implementation of Gradient Descent include \texttt{animation}. The \texttt{gsl} package in the \texttt{multimin()} function provides an implementation of Steepest Descent that makes use of the GNU Scientific Library \cite{Hankin2011}.

% References: Deeper understanding
% The references element description includes a listing of both primary sources of information about the technique as well as useful introductory sources for novices to gain a deeper understanding of the theory and application of the technique. The description consists of hand-selected reference material including books, peer reviewed conference papers, journal articles, and potentially websites. A bullet-pointed structure is suggested.
\subsection{References}
% What are the primary sources for a technique?
% What are the suggested reference sources for learning more about a technique?

% primary sources
\subsubsection{Primary Sources}
% seminal
The method of Steepest Descent can be traced back to Cauchy in 1847 who proposed the use of the gradient to minimize a system of simultaneous equations \cite{Cauchy1847} (French). Gradient Descent may be considered a relaxation of this original method.
% early extensions
There have been many extensions of the method. Curry provides an early extension that seeks the first stable point on the line following the gradient \cite{Curry1944}. Spang provided a good early review of minimization methods providing context for Steepest Descent \cite{Spang1962}.

% more info
\subsubsection{More Information}
Gradient Descent is a cornerstone of Machine Learning methods and is the go-to method for optimizing the cost or loss functions at the core of many modern algorithms. As such, a description can be found in any good text on Artificial Intelligence, Machine Learning, or Numerical Methods.


% END
\putbib\end{bibunit}
\newpage\begin{bibunit}% The Clever Algorithms Project: http://www.CleverAlgorithms.com
% (c) Copyright 2011 Jason Brownlee. Some Rights Reserved. 
% This work is licensed under a Creative Commons Attribution-Noncommercial-Share Alike 2.5 Australia License.

% Name
% The algorithm name defines the canonical name used to refer to the technique, in addition to common aliases, abbreviations, and acronyms. The name is used in terms of the heading and sub-headings of an algorithm description.
\section{Conjugate Gradient Method} 
\label{sec:conjugate_gradient}
\index{Conjugate Gradient Method}

% other names
% What is the canonical name and common aliases for a technique?
% What are the common abbreviations and acronyms for a technique?
\emph{Conjugate Gradient Method.}

% Taxonomy: Lineage and locality
\subsection{Taxonomy}
% To what fields of study does a technique belong?
Conjugate Gradient Method is a first-order derivative optimization method for multidimensional nonlinear unconstrained functions.
% What are the closely related approaches to a technique?
It is related to other first-order derivative optimization algorithms such as Gradient Descent and Steepest Descent.


% Strategy: Problem solving plan
% The strategy is an abstract description of the computational model. The strategy describes the information processing actions a technique shall take in order to achieve an objective. The strategy provides a logical separation between a computational realization (procedure) and a analogous system (metaphor). A given problem solving strategy may be realized as one of a number specific algorithms or problem solving systems. The strategy description is textual using information processing and algorithmic terminology.
\subsection{Strategy}
% What is the information processing objective of a technique?
% What is a techniques plan of action?

Uses conjugate directions rather than local gradients to move downhill towards the function minima. 

% Does it use a line search for alpha?

% help me use this technique
\subsection{Overview}

% what it is good at
\subsubsection{Features}

\begin{itemize}
	\item A learning rate does not need to be specified.
	\item Does not store a matrix (like BFGS) and there fore may be successful (less memory) on larger problems? (verify?)
\end{itemize}

% what it is not good at
\subsubsection{Limitations}

\begin{itemize}
	\item More fragile than BFGS? (verify?)
\end{itemize}

% sample script in R
\subsection{Code Listing}
% listing
Listing~\ref{stats_conjugate_gradient} provides a code listing of the Conjugate Gradient method in R solving a two-dimensional nonlinear unconstrained optimization function.
% algorithm and package
The example uses the {optim()} function in the \texttt{stats} core package configured to use the Conjugate Gradient method as ``CG''. The method supports three update methods: Fletcher-Reeves update (default), Polak-Ribiere update, and Beale-Sorenson update. For more information on the function, type \texttt{?optim}.
% problem
The test problem is the Rosenbrock function in two-dimensions where the optimum is at $x=1, y=1$. The starting position for the algorithm is taken as a random point $x,y \in [-3,3]$.  The gradient for this function is specified and is used by the optimization method.

% TODO consider capturing algorithm progression (using par)

\lstinputlisting[firstline=7,language=r,caption={Example of Conjugate Gradient in R using the \texttt{optim()} function in the \texttt{stats} core package.}, label=stats_conjugate_gradient]{../src/algorithms/optimization/stats_conjugate_gradient.R}

% other packages ???


% References: Deeper understanding
% The references element description includes a listing of both primary sources of information about the technique as well as useful introductory sources for novices to gain a deeper understanding of the theory and application of the technique. The description consists of hand-selected reference material including books, peer reviewed conference papers, journal articles, and potentially websites. A bullet-pointed structure is suggested.
\subsection{References}
% What are the primary sources for a technique?
% What are the suggested reference sources for learning more about a technique?

% primary sources
\subsubsection{Primary Sources}
% seminal
The Conjugate Gradient method was proposed by Fletcher and Reeves in 1964 \cite{Fletcher1964}.

% more info
\subsubsection{More Information}

See \cite{Shewchuk1994} for a good overview.


% END
\putbib\end{bibunit}
\newpage\begin{bibunit}% The Clever Algorithms Project: http://www.CleverAlgorithms.com
% (c) Copyright 2011 Jason Brownlee. Some Rights Reserved. 
% This work is licensed under a Creative Commons Attribution-Noncommercial-Share Alike 2.5 Australia License.

% Name
% The algorithm name defines the canonical name used to refer to the technique, in addition to common aliases, abbreviations, and acronyms. The name is used in terms of the heading and sub-headings of an algorithm description.
\section{Broyden-Fletcher-Goldfarb-Shanno Method} 
\label{sec:bfgs}
\index{Broyden-Fletcher-Goldfarb-Shanno Method}
\index{BFGS Method}
\index{Variable Metric Method}
\index{L-BFGS}
\index{Limited Memory BFGS}
\index{BFGS-B}

% other names
% What is the canonical name and common aliases for a technique?
% What are the common abbreviations and acronyms for a technique?
\emph{Broyden-Fletcher-Goldfarb-Shanno Method, BFGS Method, Variable Metric Method.}

% Taxonomy: Lineage and locality
\subsection{Taxonomy}
% To what fields of study does a technique belong?
BFGS is an optimization method for multidimensional nonlinear unconstrained functions.
% What are the closely related approaches to a technique?
BFGS belongs to the family of Quasi-Newton (Variable Metric) optimization methods that make use of both first-derivative (gradient) and second-derivative (Hessian matrix) based information of the function being optimized. It is related to other second-derivative optimization methods such as Newton's Method (Newton-Raphson Method), Gauss-Newton Method, Levenberg-Marquardt Method. 
% extensions
Two popular extension of BFGS is L-BFGS (Limited Memory BFGS) which has lower memory resource requirements and BFGS-B (Box BFGS) which imposes box constraints on the method.

% Strategy: Problem solving plan
% The strategy is an abstract description of the computational model. The strategy describes the information processing actions a technique shall take in order to achieve an objective. The strategy provides a logical separation between a computational realization (procedure) and a analogous system (metaphor). A given problem solving strategy may be realized as one of a number specific algorithms or problem solving systems. The strategy description is textual using information processing and algorithmic terminology.
\subsection{Strategy}
% What is the information processing objective of a technique?
% What is a techniques plan of action?


% Heuristics: Usage guidelines
% The heuristics element describe the commonsense, best practice, and demonstrated rules for applying and configuring a parameterized algorithm. The heuristics relate to the technical details of the techniques procedure and data structures for general classes of application (neither specific implementations not specific problem instances). The heuristics are described textually, such as a series of guidelines in a bullet-point structure.
\subsection{Heuristics}
% What are the suggested configurations for a technique?
% What are the guidelines for the application of a technique to a problem instance?

\begin{itemize}
	\item It requires a relatively large memory footprint, as it maintains an $n*n$ Hessian matrix, where $n$ is the number of variables. This is a limitation on the methods scalability.
\end{itemize}

% sample script in R
\subsection{Code Listing}
% listing
Listing~\ref{stats_bfgs} provides a code listing of the BFGS method in R solving a two-dimensional nonlinear unconstrained optimization function.
% algorithm and package
The example uses the {optim()} function in the \texttt{stats} core package configured to use the BFGS method. For more information on the function, type \texttt{?optim}.
% problem
The test problem is the Rosenbrock function in two-dimensions where the optimum is at $x=1, y=1$. The starting position for the algorithm is taken as a random point $x,y \in [-3,3]$. The gradient for this function is specified and is used by the optimization method.

% TODO consider capturing algorithm progression (using par)

\lstinputlisting[firstline=7,language=r,caption={Example of BFGS in R using the \texttt{optim()} function in the \texttt{stats} core package.}, label=stats_bfgs]{../src/algorithms/optimization/stats_bfgs.R}

% other packages
The \texttt{optim()} function in the \texttt{stats} core package also offers an implementation of BFGS with both the Limited Memory (L-BFGS) and box constraint (BFGS-B) extensions based on the FORTRAN implementation described by Zhu et~al. \ \cite{Zhu1997}.

% References: Deeper understanding
% The references element description includes a listing of both primary sources of information about the technique as well as useful introductory sources for novices to gain a deeper understanding of the theory and application of the technique. The description consists of hand-selected reference material including books, peer reviewed conference papers, journal articles, and potentially websites. A bullet-pointed structure is suggested.
\subsection{References}
% What are the primary sources for a technique?
% What are the suggested reference sources for learning more about a technique?

% primary sources
\subsubsection{Primary Sources}
% seminal 
The BFGS method was published by four authors at the same time in 1970: Broyden \cite{Broyden1970}, Fletcher \cite{Fletcher1970}, Goldfarb \cite{Goldfarb1970}, and Shanno \cite{Shanno1970}.

% more info
\subsubsection{More Information}




% END
\putbib\end{bibunit}

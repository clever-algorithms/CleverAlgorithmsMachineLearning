% The Clever Algorithms Project: http://www.CleverAlgorithms.com
% (c) Copyright 2010 Jason Brownlee. Some Rights Reserved. 
% This work is licensed under a Creative Commons Attribution-Noncommercial-Share Alike 2.5 Australia License.

% This is a chapter

\renewcommand{\bibsection}{\subsection{\bibname}}
\begin{bibunit}

\chapter{Regularization}
\label{ch:regularization}
\index{Regularization}

\section{Overview}
This chapter describes Regularization.

% what is regularization?
\subsection{Description}
\index{Regularization!Description}
% what is it
Regularization is generally a method for fine tuning of the bias-variance trade-off of a model in the context of a training dataset by penalizing model complexity.
% how does it generally work
A common example in linear models is to modify the fit equation and/or loss function to include a penalty for the number or absolute magnitude of summed model coefficients. Such modifications would then favoring models with fewer coefficients or lower a lower absolute magnitude of summed coefficients. The effect is simpler models and the removal of parameters as coefficient values move toward zero. A validation dataset (separate from the training dataset) or cross validation is commonly used to assess the predictive capability of the model during the regularization process.

% abstraction
In this regard, regularization may be conceptualized as feature selection or model selection that favors fewer features or less complex models. The reduction of complexity or shrinking of the model can reduce variance in the model and improve the models ability to generalize (address the model over-fitting the training dataset). 
% when to use
Regularization is useful when the risk of overfitting is high, such as when you have small amount of training data, few observations of a given type, or when you have more features than you do observations.

% examples of methods
\subsection{Taxonomy}
\index{Regularization!Taxonomy}
A useful way to conceptualize the study of Regularization methods is to consider the type of penalty that is added to the cost function.

\index{LASSO}
\index{Basis Pursuit}
\index{Group LASSO}
\index{Blockwise Sparse Regression}
\index{Fused LASSO}
\index{Graphical LASSO}
\index{Ridge Regression}
\index{Elastic Net}
\index{Smoothly Clipped Absolute Deviation}
\index{SCAD}
\index{$L_1$-norm Penalty}
\index{$L_2$-norm Penalty}
\index{Bridge Regression}
\index{Non-negative Garrote}
\index{Stepwise Regression}
\index{Subset Selection}
\begin{description}
	\item[$L_1$-norm Penalty] This is the sum of the absolute values in a regression model or the $L_1$-norm of the coefficient vector. Examples include the Dantzig selector, and the LASSO (a.k.a Basis Pursuit) and extensions to the LASSO such as Group LASSO, Blockwise Sparse Regression (BSR), Fused LASSO, Graphical LASSO.
	\item[$L_2$-norm Penalty] This is the squared sum of the absolute values in a regression model or the $L_2$-norm of the coefficient vector. Examples include Ridge Regression.
	\item[Hybrid of $L_1$-norm and $L_2$-norm Penalty] These are methods that makes use of both types of penalty or combine them in some way. Examples include the Elastic Net and Smoothly Clipped Absolute Deviation (SCAD) penalty.
	\item[$L_q$-norm Penalty] These are a generalization of the $L_1$ and $L_2$ norm penalties. Examples include Bridge Regression.
	\item[Other] These are regularization methods that don't fit into the chosen taxonomy. Examples include the Non-negative Garrote. Other methods related to regularization that are commonly directly compared in simulation studies include Stepwise Regression and Subset Selection.
\end{description}

Methods such as the LASSO and Elastic Net pose difficult function optimization problems. Much research goes into the development of efficient algorithmic solutions to these objective functions. Examples include Least Angle Regression (LARS) and Cyclical Coordinate Descent.

You can regularize many different model types. For example, it is common to prepend ``Regularized'' to the names models that make use of a regularization method such as Regularized Linear Regression and Regularized Support Vector Machine. It is also common for regularization methods to be described and analyzed in the context of simple regression models such as Linear Regression.

% further reading
\subsection{Further Reading}
\index{Regularization!Further Reading}
Regularization as a subject of study in its own right is a relatively recent development. There are few if any texts dedicated to it. Some Machine Learning texts provide a treatment of the subject in the context of modifications to regression models.

Nevertheless, Hastie et~al.\ provide a detailed treatment of regularization methods (\cite{Hastie2009}, Chapter 3) and also go on to describe more sophisticated methods (\cite{Hastie2009}, Chapter 5).

\putbib
\end{bibunit}

\newpage\begin{bibunit}% The Clever Algorithms Project: http://www.CleverAlgorithms.com
% (c) Copyright 2011 Jason Brownlee. Some Rights Reserved. 
% This work is licensed under a Creative Commons Attribution-Noncommercial-Share Alike 2.5 Australia License.

% Name
% The algorithm name defines the canonical name used to refer to the technique, in addition to common aliases, abbreviations, and acronyms. The name is used in terms of the heading and sub-headings of an algorithm description.
\section{Ridge Regression} 
\label{sec:ridge}
\index{Ridge Regression}
\index{Tikhonov Regularization}
\index{Tikhonov-Miller Method}
\index{Phillips-Twomey Method}
\index{Constrained Linear Inversion}
\index{Linear Regularization}

% other names
% What is the canonical name and common aliases for a technique?
% What are the common abbreviations and acronyms for a technique?
\emph{Ridge Regression, Tikhonov Regularization, Tikhonov-Miller Method, Phillips-Twomey Method, Constrained Linear Inversion, Linear Regularization}

% Taxonomy: Lineage and locality
% The algorithm taxonomy defines where a techniques fits into the field, both the specific subfields of Computational Intelligence and Biologically Inspired Computation as well as the broader field of Artificial Intelligence. The taxonomy also provides a context for determining the relation- ships between algorithms. The taxonomy may be described in terms of a series of relationship statements or pictorially as a venn diagram or a graph with hierarchical structure.
\subsection{Taxonomy}
% To what fields of study does a technique belong?
% What are the closely related approaches to a technique?
Ridge Regression is a Regression algorithm.

It is a linear regularization method.
used with least squares regression and logistic regression !?


% Strategy: Problem solving plan
% The strategy is an abstract description of the computational model. The strategy describes the information processing actions a technique shall take in order to achieve an objective. The strategy provides a logical separation between a computational realization (procedure) and a analogous system (metaphor). A given problem solving strategy may be realized as one of a number specific algorithms or problem solving systems. The strategy description is textual using information processing and algorithmic terminology.
\subsection{Strategy}
% What is the information processing objective of a technique?
% What is a techniques plan of action?

used for Regularization
find coefficients via optimization - unconstrained multipler
assumes that coefficients after normalization are not very large

Bayesian understanding of the method
Restricted Least Squares understanding of the method 

see \url{http://cran.r-project.org/doc/contrib/Faraway-PRA.pdf}

% help me use this technique
\subsection{Overview}

% what it is good at
\subsubsection{Features}

\begin{itemize}
	\item Ridge regression result in small coefficients which may be considered a less complex model.
	\item The method is appropriate when other methods (such as least squares) appear to be unstable.
	\item Appropriate when the design matrix is collinear. (?)
	\item The ridge constant $\lambda$ is typically selected in the range $[0,1]$, where $\lambda = 0$ corresponds to a least squares regression.
\end{itemize}

% what it is not good at
\subsubsection{Limitations}

\begin{itemize}
	\item The coefficients prepared by Ridge Regression is biased. The cost function is the square of the bias plus the variance. An improvement can be achieved by a large drop in variance at the expense of bias.
\end{itemize}


% sample script in R
\subsection{Code Listing}
% listing
Listing~\ref{mass_ridge_regression} provides a code listing Ridge Regression method in R to find a line of best fit for a two-dimensional data set.
% algorithm and package
The example uses the {lm.ridge()} function in the \texttt{MASS} core package which is responsible for fitting linear models using Ridge Regression.
% problem
%The test problem is a two-dimensional dataset of 50 samples, where the x-values are drawn from a uniformly random distribution $x \in [0,10]$ and y values are the x value plus a value drawn from a normally random distribution with a mean of 0 and a standard deviation of 1.

% how do make use of a resulting lambda?
% how can you make use of a resulting model to make predictions?


\lstinputlisting[firstline=7,language=r,caption={Example of Ridge Regression in R using the \texttt{lm.ridge()} function in the \texttt{MASS} package.}, label=mass_ridge_regression]{../src/algorithms/regression/mass_ridge_regression.R}

% other packages ?
Other packages that provide an implementation of Ridge Regression include \texttt{parcor}.


% References: Deeper understanding
% The references element description includes a listing of both primary sources of information about the technique as well as useful introductory sources for novices to gain a deeper understanding of the theory and application of the technique. The description consists of hand-selected reference material including books, peer reviewed conference papers, journal articles, and potentially websites. A bullet-pointed structure is suggested.
\subsection{References}
% What are the primary sources for a technique?
% What are the suggested reference sources for learning more about a technique?

% primary sources
\subsubsection{Primary Sources}


% more info
\subsubsection{More Information}



% END
\putbib\end{bibunit}
\newpage\begin{bibunit}% The Clever Algorithms Project: http://www.CleverAlgorithms.com
% (c) Copyright 2011 Jason Brownlee. Some Rights Reserved. 
% This work is licensed under a Creative Commons Attribution-Noncommercial-Share Alike 2.5 Australia License.

% Name
% The algorithm name defines the canonical name used to refer to the technique, in addition to common aliases, abbreviations, and acronyms. The name is used in terms of the heading and sub-headings of an algorithm description.
\section{LASSO} 
\label{sec:lasso}
\index{LASSO, Least Absolute Shrinkage and Selection Operator}

% other names
% What is the canonical name and common aliases for a technique?
% What are the common abbreviations and acronyms for a technique?
\emph{LASSO}

% Taxonomy: Lineage and locality
% The algorithm taxonomy defines where a techniques fits into the field, both the specific subfields of Computational Intelligence and Biologically Inspired Computation as well as the broader field of Artificial Intelligence. The taxonomy also provides a context for determining the relation- ships between algorithms. The taxonomy may be described in terms of a series of relationship statements or pictorially as a venn diagram or a graph with hierarchical structure.
\subsection{Taxonomy}
% To what fields of study does a technique belong?
% What are the closely related approaches to a technique?
LASSO is a Regularization algorithm.
model selection technique

Least Angle Regression (LAR) algorithm for solving the Lasso

grouped lasso is an extension?
Dantzig selector is an extension?
elastic net is an extension?
generalized elastic net is an extension?
graphical lasso is an extension?




% Strategy: Problem solving plan
% The strategy is an abstract description of the computational model. The strategy describes the information processing actions a technique shall take in order to achieve an objective. The strategy provides a logical separation between a computational realization (procedure) and a analogous system (metaphor). A given problem solving strategy may be realized as one of a number specific algorithms or problem solving systems. The strategy description is textual using information processing and algorithmic terminology.
\subsection{Strategy}
% What is the information processing objective of a technique?
% What is a techniques plan of action?

penalized least squares

lasso is for linear regression models?

LARS algorithm

uses the L1-penalty, the lasso does both continuous shrinkage and automatic variable selection simultaneously

imposes a bound on the absolute sum of the coefficients

\subsection{Overview}

% what it is good at
\subsubsection{Features}

\begin{itemize}
	\item results in a sparse representation (fewer variables)
\end{itemize}

% what it is not good at
\subsubsection{Limitations}

\cite{Zou2005} mentions 3 limitations of LASSO

\begin{itemize}
	\item not good at p>>n (more attributes than instances)
\end{itemize}


% sample script in R
\subsection{Code Listing}
% listing
Listing~\ref{lars_lasso_regression} provides a code listing LASSO method in R.
% algorithm and package
The example uses the \texttt{lars()} function in the \texttt{lars} core package. The \texttt{lars} package provides the Least Angle Regression, Lasso, and the Forward Stagewise algorithms for regularization \cite{Hastie2011}.
% problem


% TODO provide a better problem with real variable selection

\lstinputlisting[firstline=7,language=r,caption={Example of LASSO in R using the \texttt{lars()} function in the \texttt{lars} package.}, label=lars_lasso_regression]{../src/algorithms/regularization/lars_lasso_regression.R}

% other packages
Other packages provide implementations of the Lasso penalty method.
The \texttt{glmnet} package provides the Lasso and elastic-net regularization for generalized linear models \cite{Friedman2011}.
The \texttt{grplasso} package provides an the Group Lasso penalty method for fitting models \cite{Meier2009}.
The \texttt{grpreg} package provides lasso regularization with grouped covariates \cite{Brehen2011}.

% lmmlasso package?!?


% References: Deeper understanding
% The references element description includes a listing of both primary sources of information about the technique as well as useful introductory sources for novices to gain a deeper understanding of the theory and application of the technique. The description consists of hand-selected reference material including books, peer reviewed conference papers, journal articles, and potentially websites. A bullet-pointed structure is suggested.
\subsection{References}
% What are the primary sources for a technique?
% What are the suggested reference sources for learning more about a technique?

% primary sources
\subsubsection{Primary Sources}

Proposed by Tibshirani \cite{Tibshirani1996}.

% more info
\subsubsection{More Information}



% END
\putbib\end{bibunit}
\newpage\begin{bibunit}% The Clever Algorithms Project: http://www.CleverAlgorithms.com
% (c) Copyright 2011 Jason Brownlee. Some Rights Reserved. 
% This work is licensed under a Creative Commons Attribution-Noncommercial-Share Alike 2.5 Australia License.

% Name
% The algorithm name defines the canonical name used to refer to the technique, in addition to common aliases, abbreviations, and acronyms. The name is used in terms of the heading and sub-headings of an algorithm description.
\section{Elastic Net} 
\label{sec:elasticnet}
\index{Elastic-Net}

% other names
% What is the canonical name and common aliases for a technique?
% What are the common abbreviations and acronyms for a technique?
\emph{Elastic Net, Elastic-Net, Na\"ive Elastic Net}

% Taxonomy: Lineage and locality
% The algorithm taxonomy defines where a techniques fits into the field, both the specific subfields of Computational Intelligence and Biologically Inspired Computation as well as the broader field of Artificial Intelligence. The taxonomy also provides a context for determining the relation- ships between algorithms. The taxonomy may be described in terms of a series of relationship statements or pictorially as a venn diagram or a graph with hierarchical structure.
\subsection{Taxonomy}
% To what fields of study does a technique belong?
Elastic Net is a Regularization method for Multiple Linear Regression models, and can be generalized to other machine learning methods.
% What are the closely related approaches to a technique?
It is related to other Regularization methods such as LASSO and Ridge Regression.

% Strategy: Problem solving plan
% The strategy is an abstract description of the computational model. The strategy describes the information processing actions a technique shall take in order to achieve an objective. The strategy provides a logical separation between a computational realization (procedure) and a analogous system (metaphor). A given problem solving strategy may be realized as one of a number specific algorithms or problem solving systems. The strategy description is textual using information processing and algorithmic terminology.
\subsection{Strategy}
% What is the information processing objective of a technique?
The information processing objective of the Elastic Net is to promote a parsimonious model.
% What is a techniques plan of action?
This is achieved by adding a penalty term that is a sum of the term used for Ridge Regression (squared sum of the coefficient values or $L_2$-norm) and the term used in the LASSO method (sum of absolute coefficient values or the $L_1$-norm). This penalty term is added to the models cost function with a penalty weight parameter ($\alpha$) and a shrinkage parameter $t$ that imposes a linear constraint on the term. 
It was proposed to address the limitations in the LASSO when Ridge Regression was demonstrated to perform better. 

Na\"ive Elastic Net is a two-stage procedure, first for each 

% optimization solutions
A modification of the Least Angle Regression (LARS) algorithm was proposed to solve the Elastic Net objective function called Least Angle Regression Elastic Net (LARS-EN).

% Heuristics: Usage guidelines
% The heuristics element describe the commonsense, best practice, and demonstrated rules for applying and configuring a parameterized algorithm. The heuristics relate to the technical details of the techniques procedure and data structures for general classes of application (neither specific implementations not specific problem instances). The heuristics are described textually, such as a series of guidelines in a bullet-point structure.
\subsection{Heuristics}
% What are the suggested configurations for a technique?
% What are the guidelines for the application of a technique to a problem instance?

\begin{itemize}
	\item It is expected to be useful in problems with a large number of predictors $p$ and a smaller number of observations $n$, $p>n$ problems.
	\item When the penalty weighting is 1 ($\alpha=1$), Na\"ive Elastic Net behaves like Ridge Regression. $\alpha$ is typically $\in \{0.1)$
	\item It can perform shrinkage of coefficients and variable selection simultaneously in the same way that the LASSO can.
	\item It has been demonstrated to outperform the LASSO while achieving a similarly sparse models.
	\item It encourages a grouping effect, where strongly correlated predictors tend to be in or out of the model together, a feature that had to be explicitly added to the LASSO in Group LASSO.
\end{itemize}

% sample script in R
\subsection{Code Listing}
% listing
Listing~\ref{glmnet_elastic_net} provides a code listing Elastic Net method in R.
% algorithm and package
The example uses the \texttt{glmnet()} function in the \texttt{glmnet} core package. The demonstration provides an example of a logistic regression with regularization.
% problem

% TODO provide a better problem with real variable selection

\lstinputlisting[firstline=7,language=r,caption={Example of Elastic Net in R using the \texttt{glmnet()} function in the \texttt{glmnet} package.}, label=glmnet_elastic_net]{../src/algorithms/regularization/glmnet_elastic_net.R}

% other packages
% elasticnet package?


% References: Deeper understanding
% The references element description includes a listing of both primary sources of information about the technique as well as useful introductory sources for novices to gain a deeper understanding of the theory and application of the technique. The description consists of hand-selected reference material including books, peer reviewed conference papers, journal articles, and potentially websites. A bullet-pointed structure is suggested.
\subsection{References}
% What are the primary sources for a technique?
% What are the suggested reference sources for learning more about a technique?

% primary sources
\subsubsection{Primary Sources}
% seminal
The Elastic-Net was described by Zou and Hastie to address short-comings in LASSO by combining the penalty functions of Ridge Regression and the LASSO \cite{Zou2005}. Both a Na\"ive Elastic Net and the Elastic Net method proper were described as was a solution to the objective function called Least Angle Regression Elastic Net (LARS-EN).

% more info
\subsubsection{More Information}



% END
\putbib\end{bibunit}
